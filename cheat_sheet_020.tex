\documentclass{article}

\usepackage[utf8]{inputenc}					% gestion des accents (source)
\usepackage[T1]{fontenc}					% gestion des accents (PDF)
\usepackage[frenchb]{babel}				    % gestion du français

\usepackage{textcomp}						% caractères additionnels
\usepackage{mathtools,amssymb,amsthm}		% packages de l'AMS + mathtools
\usepackage{lmodern}						% police de caractère
\usepackage{stmaryrd}						% symboles supplémentaires
\usepackage{csquotes}
\usepackage{empheq}							% pour encadrer.

\usepackage[landscape]{geometry}						% gestion des marges
\geometry{margin=.5cm}

\usepackage{graphicx}						% gestion des images
\usepackage{xcolor}							% gestion des couleurs
\usepackage{array}							% gestion améliorée des tableaux
\usepackage{multirow}						% gestion améliorée des colonnes

\usepackage[framemethod=tikz]{mdframed}		% mise en page
\usepackage{calc}							% syntaxe naturelle pour les calculs
\usepackage[pagestyles]{titlesec}			% pour les sections
\usepackage{titletoc}						% pour la table des matières
\usepackage{fancyhdr}						% pour les en-têtes
\usepackage{wrapfig}

\usepackage{tikz, pgf}
\usepackage{tikz-3dplot}
\usepackage{tkz-euclide}
\usepackage{pgfplots}						% tracer des courbes
\usepackage{pgfplotstable}

\usepackage{hyperref}						% permet de mettre des url cliquables
\usepackage{listings}						% permet de mettre du code

\usepackage{colortbl}
\usepackage{makecell}

\usepackage{multicol}

\definecolor{LightGray}{gray}{0.8}
\setlength\parindent{0cm}




\begin{document}
\begin{multicols}{2}
\section*{Préambule}
\begin{tabular}{|l|l|}
	\arrayrulecolor{LightGray}
	\rowcolor{LightGray} Comande & Effet\\\hline
	\verb!\title{Titre}! & Titre du document \\\hline
	\verb!\author{Auteur}! & Auteur du document \\\hline
	\verb!\date{date}! & Date (accèpte l'argument \verb!\today!)\\\hline
	\verb!\maketitle! & Affiche le titre, auteur et date\\\hline
	\verb!\tableofcontents! & Table des matières automatique\\\hline
	\verb!\listoffigures! & Table des figures automatique\\\hline
\end{tabular}


\section*{Paquets utiles}
\begin{tabular}{|l|l|}
	\arrayrulecolor{LightGray}\hline
	\rowcolor{LightGray} Paquet & Utilisation\\\hline
	\verb!\usepackage[utf8]{inputenc}! & Gestion des accents (source)\\\hline
	\verb!\usepackage[T1]{fontenc}! & Gestion des accents (PDF)\\\hline
	\verb!\usepackage[frenchb]{babel}! & Gestion du français\\\hline
\end{tabular}
\ \\\ \\\ \\
\begin{tabular}{|l|l|}
	\arrayrulecolor{LightGray}\hline
	\rowcolor{LightGray} Paquet & Utilisation\\\hline

	\verb!\usepackage{mathtools,amssymb,amsthm}! & Mathématiques standards\\\hline
	\verb!\usepackage{textcomp}!& Symboles supplémentaires\\\hline
	\verb!\usepackage[margin=1in]{geometry}! & Gestion des marges\\\hline
	\verb!\usepackage{graphicx}! & Gestion des images\\\hline
	\verb!\usepackage{xcolor}! & Gestion des colonnes\\\hline
	\verb!\usepackage{array}! & Gestion améliorée des tableaux\\\hline
	\verb!\usepackage{multirow}! & Gestion améliorée des colonnes \\\hline
	\verb!\usepackage{fancyhdr}! & Gestion des en-têtes \\\hline
	\verb!\usepackage{wrapfig}! & Interaction texte/image\\\hline
	\verb!\usepackage{tikz,pgfplots}! & Gestion des graphes et courbes\\\hline
	\verb!\usepackage{hyperref}! & Création de liens cliquables\\\hline
	\verb!\usepackage{listings}! & Ecriture de code\\\hline
	\verb!\usepackage{minted}! & Autre facon d'écrire du code\\\hline
	\verb!\usepackage{physics}! & Symboles physiques \\\hline
\end{tabular}


\end{multicols}

\rule{\textwidth}{.5pt}


\begin{multicols}{2}[\section*{Corps de texte}]

	\begin{tabular}{|l|l|}
	\arrayrulecolor{LightGray}
	\hline
    \rowcolor{LightGray} Comande & Effet\\\hline
    \verb!\begin{document}! & \multirow{2}{*}{Environnement "Document"} \\
    \verb!\end{document} !& \\\hline
	\verb!\part{Nom partie}! & Crée une partie\\\hline
    \verb!\section{Nom section}! & Crée une section\\\hline
  	\verb!\subsection{Nom section}! & Crée une sous-section\\\hline
    \verb!\subsubsection{Nom section}! & Crée une sous-sous-section\\\hline
    \verb!\begin{itemize}! & \multirow{3}{*}{Crée une liste type bullet point}\\
    \verb!  \item! \textit{blabla}  & \\
    \verb!\end{itemize}! & \\\hline
    \verb!\begin{enumerate}! & \multirow{3}{*}{Crée une liste numérotée}\\
    \verb!  \item! \textit{blabla}  & \\
    \verb!\end{enumerate}! & \\\hline
    \verb!\begin{tabular}{l|c|r}! & \multirow{4}{*}{\begin{tabular}{l|c|r}
    blabla & bla & blabla\\\hline
    bla & blabla & bla\\
    \end{tabular}} \\
    \verb!  blabla & bla & blabla\\\hline! & \\
    \verb!  bla & blabla & bla\\! & \\
    \verb!\end{tabular}! & \\\hline
    \multicolumn{2}{|c|}{\textit{Pour tout tableau non trivial, go:} \url{www.tablesgenerator.com}}\\\hline
    \verb!\begin{center}! & \multirow{2}{*}{Environnement centré} \\
    \verb!\end{center}! & \\\hline
    \verb!\begin{flushright}! & \multirow{2}{*}{Environnement aligné à droite} \\
    \verb!\end{flushright}! & \\\hline
    \verb!\begin{flushleft}! & \multirow{2}{*}{Environnement aligné à gauche} \\
    \verb!\end{flushleft}! & \\\hline

	\end{tabular}


	\begin{tabular}{|l|l|}
    \arrayrulecolor{LightGray}
    \hline
    \rowcolor{LightGray} Commande & Effet \\\hline
    \verb!\textsc{Texte en Majuscules}! & \textsc{Texte en Majuscules} \\\hline
    \verb!\textbf{Texte en gras}! & \textbf{Texte en gras} \\\hline
	  \verb!\emph{Texte en italique}!& \emph{Texte en italique}\\\hline
    \verb!\fbox{encadré}!& \fbox{encadré}\\\hline
    \verb!\underline{souligné}!& \underline{souligné}\\\hline
    \verb!\textsuperscript{exposant}!& \textsuperscript{exposant} \\\hline
    \verb!\textsubscript{indice}!& \textsubscript{indice} \\\hline
    \verb!\verb?texte verbatim?! & \verb?texte verbatim?\\\hline
    \verb!\newline! \emph{ou bien} \verb!\\!& Retour à la ligne\\\hline
    \verb!\newpage! & Saut de page\\\hline
     \verb!\Huge  \huge  \LARGE  \Large!&\multirow{3}{*}{Tailles de police dans l'ordre} \\
     \verb!\large  \normalsize  \small!& \\
     \verb!\footnotesize  \scritpsize  \tiny!& \\\hline
     \verb!{\Large texte }{\small texte}!& {\Large texte} {\small texte}\\\hline
         \verb!\footnote{contenu de la note}! & Note de bas de page \\\hline
    \verb!\label{nom du label}! & Créer une référence \\\hline
    \verb!\ref{nom du label}! & Renvoyer à cette référence \\\hline
    \verb!\begin{figure}[!\textit{position}\verb!]! & Envronnement flottant pour\\
    \verb!\end{figure}! & images et graphiques\\\hline
    \verb!\includegraphics[width=5cm]{image}! & Affiche l'image\\\hline
    \verb!\caption{titre du flottant}! & Légende un flottant \\\hline
	\end{tabular}
\end{multicols}

\newpage
\begin{multicols}{2}
\section*{Pour faire des maths}

\begin{tabular}{|l|l|}
	\arrayrulecolor{LightGray}
	\hline
    \rowcolor{LightGray} Comande & Effet\\\hline
    \verb!$ math stuff $! & Environnement maths inline \\\hline
    \verb!\[ math stuff \]! & Environnement maths centré, à la ligne \\\hline
    \verb!\begin{equation}! & \multirow{2}{*}{Environnement "Équation"}\\
    \verb!\end{equation}! & \\\hline
    \verb!\begin{aligned}! & Alignement d'équations:\\
    \verb!  !$2\times$\verb!1 & = !$1+1$\verb! \\! & \multirow{3}{*}{$\begin{aligned}
    2\times 1 &= 1+1 \\
    &=2
    \end{aligned}$}\\
    \verb!  & = 2! & \\
    \verb!\end{aligned}! & \\\hline
    \verb!\frac{numérateur}{dénominateur}! & $\frac{numerateur}{denominateur}$ \\\hline
    \verb!x_{indice}! & $x_{indice}$ \\\hline
    \verb!x^{exposant}! & $x^{exposant}$ \\\hline
    \verb!\bigcap \bigcup \sum \prod! & $\bigcap \ \bigcup \ \sum \ \prod$ \\\hline
    \multicolumn{2}{|l|}{\textit{Note: les indices et exposants sur les opérateurs ci-dessus se placent bien}}\\\hline
    \verb!\int_{a}^{b}! & $\int_a^b$ \\\hline
    \verb!\times \div \pm \oplus \otimes! & $\times \ \div \ \pm \ \oplus \ \otimes$\\\hline
    \verb!\cos \sin \tan \ln \log \lim! & $\cos \ \sin \ \tan \ \ln \ \log \ \lim$ \\\hline
    \verb!\forall \exists \nexists! & $\forall \ \exists \ \nexists$ \\\hline
    \verb!\leftarrow \Leftarrow! & $\leftarrow \ \Leftarrow$\\\hline
    \verb!\rightarrow \Rightarrow! & $\leftarrow \ \Leftarrow$\\\hline
    \verb!\leftrightarrow \Leftrightarrow! & $\leftrightarrow \ \Leftrightarrow$\\\hline
    \verb!\nearrow \searrow! & $\nearrow \ \searrow$\\\hline
    \verb!\in \notin \subset \supset! & $\in \ \notin \ \subset \ \supset$\\\hline
    \verb!\leqslant \geqslant \ll \gg! & $\leqslant \ \geqslant \ \ll \ \gg$\\\hline
    \verb!\sim \approx \simeq \neq! & $\sim \ \approx \ \simeq \ \neq$\\\hline
    \verb!\partial \nabla \gamma \Gamma! & $\partial \ \hbar \ \gamma \ \Gamma$ (pareil pour toute lettre grecque)\\\hline
    \verb!\infty \propto \hbar \emptyset! & $\infty \ \propto \ \hbar \ \emptyset$\\\hline
\end{tabular}


\section*{Pour aller plus loin}

\subsection*{Edition locale}
Pour pouvoir écrire sur son propre PC sans internet, pour Windows nous te conseillons \href{https://miktex.org/}{MiKTeX} qui est installé avec son propre éditeur. Sous Linux, Texlive est un package classique.

Pour un éditeur plus modulable, nous te conseilons \href{https://atom.io/}{Atom} ou \href{https://www.sublimetext.com/}{Sublime Text}, qui fonctionnent avec MiKTeX et Texlive.

\subsection*{Inclure des images}

Pour inclure une image il faut le paquet : \verb?\usepackage{graphicx}? et ensuite procèder ainsi : \\
\verb?\begin{figure}[?\emph{option}\verb?]? \\
\verb?\begin{center}? \\
\verb?\includegraphics[scale=? \emph{fraction} \verb?]{?\emph{chemin d'accès à l'image}\verb?}? \\
\verb?\caption{?\emph{légende}\verb?}? \\
\verb?\end{center}? \\
\verb?\end{figure}? \\

On peut utiliser \verb?\begin{figure}[H]? en option pour que l'image soit plus près de la position dans le code.

\rule{\textwidth}{.5pt}
\vspace{-.5cm}
\subsection*{Code informatique}
Il est également possible d'inclure du code dans un fichier \LaTeX \ grâce au paquet \verb?\usepackage{listings}? et aux commandes suivantes : \\

\noindent \verb?\lstset{language=? \emph{nom de langage} \verb?}? \\
\verb?\begin{lstlisting}? \\
\verb!for i in range (0: N) :! \\
\phantom{plop}X[i] = methode(X[i - 1]) \\
\verb?\end{lstlisting}? \\

\noindent Cela va inclure le code ainsi :

\lstset{language=python}
\begin{lstlisting}
for i in range (0: N):
	X[i] = methode(X[i - 1])
\end{lstlisting}

Il existe de nombreuses options pour ajouter la coloration syntaxique, encadrer le code, numéroter les lignes, chercher sur Internet est une bonne solution.

Le package \verb?\usepackage{minted}? possède une configuration de base plus colorée, mais a besoin d'une configuration Latex plus poussée.

\rule{\textwidth}{.5pt}

\subsection*{Liens utiles}
\begin{itemize}
	\item \url{tex.stackexchange.com}
	\item \url{www.sharelatex.com/learn/}
	\item \url{www.google.com}
\end{itemize}

\end{multicols}


\end{document}
